\documentclass{article}

\usepackage{amsmath}
\usepackage{bm}

\begin{document}
We go through the details of how portfolio level risk profile is derived.
In this document we will consider both simple return and logrithmic return
and their differences in deriving the conclusion.

\section{Definition of Asset/Portfolio Return}
Concepts defined in this section apply equally to both single financial 
asset and portfolio made of these assets.

For convenience, we use the following notation.

Let \(P_{i,t}\) denote price of asset $i$ at day $t$. 

Let \(n_i\) denote number of shares of asset $i$. This stays constant for
each asset respectively. 

Name \(\dfrac{P_{i,t}}{P_{i,t-1}}\) as price change factor, \(X_{i,t}\) 
for convenience

We can obtain the following two type of returns:

\begin{itemize}
    \item Simple return:
    \begin{math}
    R^{(S)}_{i,t} = \displaystyle \frac{P_{i,t}}{P_{i,t-1}} - 1
    \end{math}
    
    which implies that \(P_{i,t} = (1+R^{(S)}_{i,t})P_{i,t-1}\)
    \item logrithmic return:
    \begin{math}
    R^{(L)}_{i,t} = \displaystyle \ln{\frac{P_{i,t}}{P_{i,t-1}}}
    \end{math}
    
    which impies that \(P_{i,t} = e^{R^{(L)}_{i,t}}P_{i,t-1}\)
\end{itemize}

In exact form, we hvae 
\[
1+R^{(S)}_{i,t} = e^{R^{(L)}_{i,t}} = X_{i,t}
\]

Notice that when \(\dfrac{P_{i,t}}{P_{i,t-1}} \approx 1\), 
the above two returns are equivalent, that is 
\(R^{(S)}_{i,t} \approx R^{(L)}_{i,t}\) when \(P_{i,t} \approx P_{i,t-1}\)
for small daily price changes.

\section{Portfolio Daily Return}
\subsection{Exact Form}
Here we work our portfolio daily return using both simple return and 
logrithmic return at asset level. Given that simple return is a first order
approximation of logrithmic return at asset level, we will try to derive 
using logrithmic first:

let \(V_t\) be the asset value of day $t$, which can be written as 
\[
V_t = \sum_{i}^{N} n_i P_{i,t} = \sum_{i}^{N} n_i P_{i,t-1} e^{R^{(L)}_{i,t}}
\]
where \(N\) denotes the total number of constituent assets. 

Then the daily change of the portfolio would be deconstructed by using 
weights from the previous day

\begin{equation}
    \begin{aligned}
        X_{t} &= \dfrac{V_t}{V_{t-1}} & \text{by definition} \\
        &= \dfrac{\sum_{i}^{N} n_i P_{i,t-1} X_{i,t}}{V_{t-1}}
        = \sum_{i}^{N} \dfrac{n_i P_{i,t-1}}{V_{t-1}} X_{i,t} 
        & \text{by expansion} \\
        &= \sum_{i}^{N} w_{i,t-1} X_{i,t} & \text{weight definition} \\
        &= \sum_{i}^{N} w_{i,t-1} \dfrac{P_{i,t}}{P_{i,t-1}} 
        & \text{using price change}
        \label{eq:weighted_return}
    \end{aligned}
\end{equation}

where $w_{i,t-1}$ is the weight ratio of asset $i$ in the entire portfolio.
Interestingly this demonstrates the exact relationship between daily change ratio
of the portfolio and that of the constituent assets
where $V_{i,t}$ is the total value of asset $i$ on day $t$. Though this is an 
exact relationship between constituent price change and portfolio value change.

\subsection{Practial approximations}
Approximately, when daily changes are small, simple return and logrithmic return 
are close. Then \eqref{eq:weighted_return} can be written as
\[
1+R_t = \sum_{i}^{N} w_{i,t-1} (1+r_{i,t}) 
\]
where $R_t$ is portfolio return and $r_{i,t}$ is asset return.

Given that \(\sum_{1}^{N} w_{i,t-1} = 1\), we have 
\begin{equation}
    R_t = \sum_{i}^{N} w_{i,t-1} r_{i,t} \label{eq:linearity_of_return_sum}    
\end{equation}

This can be also written in a vector form
\begin{equation}
    R_t = \bm{w}^T_{t-1} \bm{r}_t \label{eq:linearity_of_return_vector}    
\end{equation}
where \(\bm{w}_{t-1}\) and \(\bm{r}_t\) are both $N \times 1$ vectors.

\subsection{Is weight vector approximately constant?}
Let us understand how $w_{i,t}$ evolves over time. Denote $V_{i,t}$ the value
of asset $i$ on day $t$
\[
w_{i,t} = \dfrac{V_{i,t}}{V_t} 
= \dfrac{V_{i,t} / V_{t-1}}{V_t / V_{t-1}}
= \dfrac{\dfrac{V_{i,t}}{V_{i,t-1}} \dfrac{V_{i,t-1}}{V_{t-1}}}{\dfrac{V_t}{V_{t-1}}} 
\]
where: \\
\begin{tabular}{ll} % means two columns, both are of left-aligned (denoted l)
    $\dfrac{V_{i,t}}{V_{i,t-1}}$ & is the growth of asset $i$ on day $t$, $X_{i,t}$. \\
    $\dfrac{V_{i,t-1}}{V_{t-1}}$ & is the weight of asset $i$ on day $t-1$, $w_{i,t-1}$. \\
    $\dfrac{V_t}{V_{t-1}}$ & is the growth of portfolio, see \eqref{eq:weighted_return}  
\end{tabular}
We obtain approximately when returns are small:
\begin{equation}
    \begin{aligned}
        w_{i,t} &= \dfrac{w_{i,t-1} X_{i,t} }{\sum_{j}^{N} w_{j,t-1} X_{j,t}} \\
        &\approx \dfrac{w_{i,t-1}(1+r_{i,t})}{1 + \sum_{j}^{N} w_{j,t-1} r_{j,t}} 
        & \text{replacing $X$ by $1+r$} \\
        &\approx w_{i,t-1}(1+r_{i,t}) (1 - \sum_{j}^{N} w_{j,t-1} r_{j,t}) 
        & \text{first order expansion} \\
        &=w_{i,t-1}(1+r_{i,t}) (1 - \bm{w}^T_{t-1} \bm{r}_t) 
        &\text{use vector form} \\
        &=w_{i,t-1}(1+r_{i,t}-\bm{w}^T_{t-1} \bm{r}_t) 
        &\text{ignore second order terms of $r$}
    \end{aligned}
\end{equation}
The crux is that 
\begin{equation}
    \begin{aligned}
        w_{i,t} &\approx w_{i,t-1} + O(r) &\text{where r is small}    
    \end{aligned}
\end{equation}


\end{document}