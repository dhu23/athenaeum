\documentclass{article}

\usepackage{amsmath}
\usepackage{bm}

\begin{document}
We go through the details of how portfolio level risk profile is derived 
in a portfolio that satisfies the following conditions:
\begin{itemize}
    \item No external cash flows (self-financing)
    \item No leverage change
    \item No rebalancing intraday
    \item Constituent assets and portfolio are priced at the end of each day.
\end{itemize}
In this document we will consider both simple return and log return
and their differences in deriving the conclusion.

\section{Definition of Asset/Portfolio Return}
Concepts defined in this section apply equally to both single financial 
asset and portfolio made of these assets.

For convenience, we use the following notation.

Let \(P_{i,t}\) denote price of asset $i$ at day $t$. 

Let \(n_i\) denote number of shares of asset $i$. This stays constant for
each asset respectively. 

Name \(\dfrac{P_{i,t}}{P_{i,t-1}}\) as price change factor, \(X_{i,t}\) 
for convenience

We can obtain the following two type of returns:

\begin{itemize}
    \item Simple return:
    \begin{math}
    R^{(S)}_{i,t} = \displaystyle \frac{P_{i,t}}{P_{i,t-1}} - 1
    \end{math}
    
    which implies that \(P_{i,t} = (1+R^{(S)}_{i,t})P_{i,t-1}\)
    \item log return:
    \begin{math}
    R^{(L)}_{i,t} = \displaystyle \ln{\frac{P_{i,t}}{P_{i,t-1}}}
    \end{math}
    
    which impies that \(P_{i,t} = e^{R^{(L)}_{i,t}}P_{i,t-1}\)
\end{itemize}

In exact form, we hvae 
\[
1+R^{(S)}_{i,t} = e^{R^{(L)}_{i,t}} = X_{i,t}
\]

Notice that when \(\dfrac{P_{i,t}}{P_{i,t-1}} \approx 1\), 
the above two returns are equivalent, that is 
\(R^{(S)}_{i,t} \approx R^{(L)}_{i,t}\) when \(P_{i,t} \approx P_{i,t-1}\)
for small daily price changes.

\section{Portfolio Daily Return}
\subsection{Exact Form}

let \(V_t\) be the asset value of day $t$, which can be written as 
\[
V_t = \sum_{i}^{N} n_i P_{i,t} = \sum_{i}^{N} n_i P_{i,t-1} e^{R^{(L)}_{i,t}}
\]
where \(N\) denotes the total number of constituent assets. 

Then the daily change of the portfolio would be deconstructed by using 
weights from the previous day

\begin{equation}
    \begin{aligned}
        X_{t} &= \dfrac{V_t}{V_{t-1}} & \text{by definition} \\
        &= \dfrac{\sum_{i}^{N} n_i P_{i,t-1} X_{i,t}}{V_{t-1}}
        = \sum_{i}^{N} \dfrac{n_i P_{i,t-1}}{V_{t-1}} X_{i,t} 
        & \text{by expansion} \\
        &= \sum_{i}^{N} w_{i,t-1} X_{i,t} & \text{weight definition} \\
        &= \sum_{i}^{N} w_{i,t-1} \dfrac{P_{i,t}}{P_{i,t-1}} 
        & \text{using price change}
        \label{eq:weighted_return}
    \end{aligned}
\end{equation}

where $w_{i,t-1}$ is the weight ratio of asset $i$ in the entire portfolio.
Interestingly this demonstrates the exact relationship between daily change ratio
of the portfolio and that of the constituent assets
where $V_{i,t}$ is the total value of asset $i$ on day $t$. Though this is an 
exact relationship between constituent price change and portfolio value change.

\subsection{Practial Approximations}
Approximately, when daily changes are small, simple return and log return 
are close. Then \eqref{eq:weighted_return} can be written as
\[
1+R_t = \sum_{i}^{N} w_{i,t-1} (1+r_{i,t}) 
\]
where $R_t$ is portfolio return and $r_{i,t}$ is asset $i$ return on day $t$.

Given that \(\sum_{1}^{N} w_{i,t-1} = 1\), we have 
\begin{equation}
    R_t = \sum_{i}^{N} w_{i,t-1} r_{i,t} \label{eq:linearity_of_return_sum}    
\end{equation}

This can be also written in a vector form
\begin{equation}
    R_t = \bm{w}^T_{t-1} \bm{r}_t \label{eq:linearity_of_return_vector}    
\end{equation}
where \(\bm{w}_{t-1} = (w_{1,t-1}, w_{2,t-1},...,w_{N,t-1})^T\) and 
\(\bm{r}_t=(r_{1,t}, r_{2,t},...,r_{N,t})^T\) are both $N \times 1$ vectors.

\subsection{Is weight vector approximately constant?}
Let us understand how $w_{i,t}$ evolves over time. Denote $V_{i,t}$ the value
of asset $i$ on day $t$
\[
w_{i,t} = \dfrac{V_{i,t}}{V_t} 
= \dfrac{V_{i,t} / V_{t-1}}{V_t / V_{t-1}}
= \dfrac{\dfrac{V_{i,t}}{V_{i,t-1}} \dfrac{V_{i,t-1}}{V_{t-1}}}{\dfrac{V_t}{V_{t-1}}} 
\]
where: \\
\begin{tabular}{ll} % means two columns, both are of left-aligned (denoted l)
    $\dfrac{V_{i,t}}{V_{i,t-1}}$ & is the growth of asset $i$ on day $t$, $X_{i,t}$. \\
    $\dfrac{V_{i,t-1}}{V_{t-1}}$ & is the weight of asset $i$ on day $t-1$, $w_{i,t-1}$. \\
    $\dfrac{V_t}{V_{t-1}}$ & is the growth of portfolio, see \eqref{eq:weighted_return}  
\end{tabular}
We obtain approximately when returns are small:
\begin{equation}
    \begin{aligned}
        w_{i,t} &= \dfrac{w_{i,t-1} X_{i,t} }{\sum_{j}^{N} w_{j,t-1} X_{j,t}} \\
        &\approx \dfrac{w_{i,t-1}(1+r_{i,t})}{1 + \sum_{j}^{N} w_{j,t-1} r_{j,t}} 
        & \text{replacing $X$ by $1+r$} \\
        &\approx w_{i,t-1}(1+r_{i,t}) (1 - \sum_{j}^{N} w_{j,t-1} r_{j,t}) 
        & \text{first order expansion} \\
        &=w_{i,t-1}(1+r_{i,t}) (1 - \bm{w}^T_{t-1} \bm{r}_t) 
        &\text{use vector form} \\
        &=w_{i,t-1}(1+r_{i,t}-\bm{w}^T_{t-1} \bm{r}_t) 
        &\text{ignore second order terms of $r$}
    \end{aligned}
\end{equation}
The crux is that the difference of weights between neighboring days is at the 
first order of $r$. 
\begin{equation}
    \begin{aligned}
        w_{i,t} &\approx w_{i,t-1} + O(r) &\text{where r is small}    
    \end{aligned}
\end{equation}
which would lead \eqref{eq:linearity_of_return_vector} to be rewritten as
\begin{equation}
    R_t = (\bm{w}^T_0 + O(r)) \bm{r}_t = \bm{w}^T_0 \bm{r}_t + O(r^2)
    \approx \bm{w}^T \bm{r}_t
    \label{eq:portfolio_return_linearity_approximation}
\end{equation}
The intuition is that though weight vector changes slowly over time, however
as far as the return of the portfolio is concerned, the impact of drift is 
at second-order. That's why in practice, the initial weight vector is used to 
perform the risk modeling

\subsection{Portfolio Risk}
The model is now equivalently a linear model if we write 
\eqref{eq:portfolio_return_linearity_approximation} using a generic 
\(\bm{w}=(w_1,w_2,w_3,...,w_N)^T\). 
Let's consider the random variable for each asset $i$ as $r_i$ (dropping $t$). 
When it is small, it is the same as the log return which is often modeled by 
normal distribution, $r_i \sim \mathcal{N}(\mu_i, \sigma^2_i)$ for analytical 
convenience. Let $R$ be the random varible for the portfolio return. On day $t$,
it takes the value of $R_t$. Similarily, let $r_i$ be the random variable for the
daily return of asset $i$. On day $t$, it takes the value of $r_{i,t}$. 
And let $\bm{r}$ be the random vector $(r_1, r_2,...,r_N)^T$. So for any given day,
$R = \bm{w}^T \bm{r}$. Across different days, $\bm{r}$ is considered i.i.d.

Now we have estabblished the relationship between the random variables, 
we can compute the expected value and variance of $R$.
\begin{equation}
    \begin{aligned}
        \operatorname{E}(R) &= \bm{w}^T \operatorname{E}(\bm{r}) \\
        \operatorname{Var}(R) 
        &= \operatorname{Var}(\bm{w}^T \bm{r}) = \bm{w}^T \bm{\Sigma}_r \bm{w}
    \end{aligned}
\end{equation}
where $\bm{\Sigma}_r$ is the covariance matrix of $r$.
We can also write that in terms of summation for variance of a linear combination
of random variables, as follows 
\begin{equation}
    \begin{aligned}
        \operatorname{Var}(R) &= \operatorname{Var}(\sum_{i}^{N} w_{i} r_{i}) \\
        &= \sum_{i}^{N} w^2_{i,t} \operatorname{Var}(r_{i}) 
        + 2 \sum_{i,j,i<j}^{N} w_i w_j \operatorname{Cov}(r_i, r_j) \\
        &= \begin{bmatrix}
            w_1 & w_2 & \cdots & w_N
        \end{bmatrix}
        \begin{bmatrix}
            \sigma^2_1 & \sigma_{1,2} & \cdots & \sigma_{1,N} \\\\
            \sigma_{2,1} & \sigma^2_2 & \cdots & \sigma_{2,N} \\\\
            \vdots & \vdots & & \vdots \\\\
            \sigma_{N,1} & \sigma_{N,2} & \cdots & \sigma^2_N
        \end{bmatrix}
        \begin{bmatrix}
            w_1 \\\\
            w_2 \\\\
            \vdots \\\\
            w_N
        \end{bmatrix} \\
        &= \bm{w}^T \bm{\Sigma}_r \bm{w}
    \end{aligned}
\end{equation}
Apparently we observe that risk (variance of return) is not additive given a 
linear relationship between portfolio return and asset returns. Correlation
between assets matters. 

\end{document}