\documentclass{article}

\usepackage{amsmath}
\usepackage{bm}

\begin{document}
We go through the details of how portfolio level risk profile is derived.
In this document we will consider both simple return and logrithmic return
and their differences in deriving the conclusion.

\section{Definition of Asset/Portfolio Return}
Concepts defined in this section apply equally to both single financial 
asset and portfolio made of these assets.

For convenience, we use the following notation.

Let \(P_{i,t}\) denote price of asset $i$ at day $t$. 

Let \(n_i\) denote number of shares of asset $i$. This stays constant for
each asset respectively. 

We can obtain the following two type of returns:

\begin{itemize}
    \item Simple return:
    \begin{math}
    R^{(S)}_{i,t} = \displaystyle \frac{P_{i,t}}{P_{i,t-1}} - 1
    \end{math}
    
    which implies that \(P_{i,t} = (1+R^{(S)}_{i,t})P_{i,t-1}\)
    \item logrithmic return:
    \begin{math}
    R^{(L)}_{i,t} = \displaystyle \ln{\frac{P_{i,t}}{P_{i,t-1}}}
    \end{math}
    
    which impies that \(P_{i,t} = e^{R^{(L)}_{i,t}}P_{i,t-1}\)
\end{itemize}

In exact form, we hvae 
\[
1+R^{(S)}_{i,t} = e^{R^{(L)}_{i,t}}
\]

Notice that when \(\frac{P_{i,t}}{P_{i,t-1}} \approx 1\), 
the above two returns are equivalent, that is 
\(R^{(S)}_{i,t} \approx R^{(L)}_{i,t}\) when \(P_{i,t} \approx P_{i,t-1}\)
for small daily price changes.

\section{Portfolio Daily Return}
Here we work our portfolio daily return using both simple return and 
logrithmic return at asset level. Given that simple return is a first order
approximation of logrithmic return at asset level, we will try to derive 
using logrithmic first:

let \(V_t\) be the asset value of day $t$, which can be written as 
\[
V_t = \sum_{i}^{N} n_i P_{i,t} = \sum_{i}^{N} n_i P_{i,t-1} e^{R^{(L)}_{i,t}}
\]
where \(N\) denotes the total number of constituent assets. 

Then the daily change of the portfolio would be deconstructed by using 
weights from the previous day

\begin{equation}
\begin{aligned}
e^{R^{(L)}_t} &= \dfrac{V_t}{V_{t-1}} & \text{by definition} \\
&= \dfrac{\sum_{i}^{N} n_i P_{i,t-1} e^{R^{(L)}_{i,t}}}{V_{t-1}}
= \sum_{i}^{N} \dfrac{n_i P_{i,t-1}}{V_{t-1}} e^{R^{(L)}_{i,t}} 
& \text{by expansion} \\
&= \sum_{i}^{N} w_{i,t-1} e^{R^{(L)}_{i,t}} & \text{weight definition}
\label{eq:weighted_return}
\end{aligned}
\end{equation}

where $w_{i,t-1}$ is the weight ratio of asset $i$ in the entire portfolio.
Interesting this demonstrate the exact relationship between daily change ratio
of the portfolio and that of the constituent assets if we replace the expoentials:
\[
\dfrac{V_t}{V_{t-1}} = \sum_{i}^{N} w_{i,t-1} \dfrac{V_{i,t}}{V_{i,t-1}}
= \sum_{i}^{N} w_{i,t-1} \dfrac{P_{i,t}}{P_{i,t-1}}
\]
where $V_{i,t}$ is the total value of asset $i$ on day $t$. Though this is an 
exact relationship between constituent price change and portfolio value change.

Approximately, when daily changes are small, simple return and logrithmic return 
are close. Then \eqref{eq:weighted_return} can be written as
\[
1+R_t = \sum_{i}^{N} w_{i,t-1} (1+r_{i,t}) 
\]
where $R_t$ is portfolio return and $r_{i,t}$ is asset return.

Given that \(\sum_{1}^{N} w_{i,t-1} = 1\), we have 
\begin{equation}
R_t = \sum_{i}^{N} w_{i,t-1} r_{i,t} \label{eq:linearity_of_return_sum}    
\end{equation}

This can be also written in a vector form
\begin{equation}
R_t = \bm{w}^T_{t-1} \bm{r}_t \label{eq:linearity_of_return_vector}    
\end{equation}
where \(\bm{w}_{t-1}\) and \(\bm{r}_t\) are both $N \times 1$ vectors.



\end{document}